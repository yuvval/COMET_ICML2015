%%%%%%%%%%%%%%%%%%%%%%%%%%%%%%%%%%%%%%%%%%%%%%%%%%%%%%%%%%%%%%%%%%
%%%%%%%% COMET ICML 2015 
%%%%%%%%%%%%%%%%%%%%%%%%%%%%%%%%%%%%%%%%%%%%%%%%%%%%%%%%%%%%%%%%%%
% Use the following line _only_ if you're still using LaTeX 2.09.
%\documentstyle[icml2015,epsf,natbib]{article}
% If you rely on Latex2e packages, like most moden people use this:
\documentclass{article}

%% COMET packages
\usepackage{amssymb}
\usepackage{amsmath}
\usepackage{mathtools}

%%%%%%%%%%%%%%%%%%%%%%%%
% use Times
\usepackage{times}
\usepackage{graphicx} % more modern
\usepackage{subfigure} 
\usepackage{amsthm}
\usepackage{natbib}
\usepackage{algorithm}
\usepackage{algorithmic}
\usepackage{hyperref}
\newcommand{\theHalgorithm}{\arabic{algorithm}}
\usepackage{icml2015} 

\begin{document} 

\twocolumn[
\icmltitle{Metric Learning using Block Coordinate Descent}

% It is OKAY to include author information, even for blind
% submissions: the style file will automatically remove it for you
% unless you've provided the [accepted] option to the icml2015
% package.

\icmlauthor{Yuval Atzmon}{yuval.atzmon@biu.ac.il}
\icmladdress{The Gonda Brain Research Center, Bar Ilan University, 52900, Israel}
\icmlauthor{Uri Shalit}{URI.SHALIT@MAIL.HUJI.AC.IL}
\icmladdress{ICNC-ELSC & Computer Science Department, The Hebrew University of Jerusalem, 91904 Jerusalem Israel

The Gonda Brain Research Center, Bar Ilan University, 52900 Ramat-Gan, Israel}
\icmlauthor{Gal Chechik}{GAL.CHECHIK@BIU.AC.IL}
\icmladdress{The Gonda Brain Research Center, Bar Ilan University, 52900, Israel}

% You may provide any keywords that you 
% find helpful for describing your paper; these are used to populate 
% the "keywords" metadata in the PDF but will not be shown in the document
\icmlkeywords{metric learning, coordinate descent, similarity learning}

\vskip 0.3in
]

%%%% COMET commands %%%%
\newcommand\todo[1]{\textbf{<ToDo:#1}!}
%\newcommand\mat[1]{\mathcal{#1}}
%\newcommand\mat[1]{\boldmath{#1}}
\newcommand\mat[1]{{#1}}
\renewcommand\vec[1]{\mathbf{#1}}
\newcommand{\T}{{}^\mathsf{T}}
\newcommand{\W}{\mat{W}}
\newcommand{\E}{\mat{E}}
\newcommand{\Hh}{\mat{H}}
\newcommand{\Pp}{\mat{P}}
\newcommand{\newW}{{\mat{W^{new}}}}
\newcommand{\eqdef}{\doteq}
\newcommand{\Rd}{\mathbb{R}^d}
\newcommand{\R}{\mathbb{R}}
\newcommand{\tL}{\tilde{L}(\W)}
\newcommand{\frobsq}[1]{{\|#1\|_F^2}}
\newcommand{\frob}[1]{{\|#1\|_F}} 
\newcommand{\ignore}[1]{}

\newcommand{\q}{{\vec{q}}}
\newcommand{\p}{{\vec{p}}}
\newcommand{\trip}{{t}}
\newcommand{\qt}{{\q_{\trip}}}
\newcommand{\pt}{{\p_{\trip}}}
\newcommand{\triplet}{(\qt, \pt^{+}, \pt^{-})}


\newcommand{\A}{A}
\newcommand{\B}{\vec{b}}
\newcommand{\C}{c}
\newcommand{\invA}{A^{-1}}

\newcommand{\grd}{\frac{\partial \tL}{\W}}
\newcommand{\grdkl}{\frac{\partial \tL}{\W_{kl}}}


\newcommand{\uscalar}{{u}_{1}}
\newcommand{\uvec}{\vec{u}_{2:d}} 
\newcommand{\Wvec}{\W_{2:d,1}}
\newcommand{\Wscalar}{\W_{1,1}}


\newtheorem{theorem}{Theorem}
\newtheorem{lemma}{Lemma}
\newtheorem{corollary}{Corollary}
\newtheorem{definition}{Definition}
\newtheorem{apptheorem}{Theorem}
\newtheorem{applemma}{Lemma}


\renewcommand{\eqref}[1]{Eq.~(\ref{#1})}
\newcommand{\figref}[1]{Fig.~\ref{#1}}
\newcommand{\secref}[1]{Sec.~\ref{#1}}
\newcommand{\tabref}[1]{Table~\ref{#1}}

%\DeclareMathOperator*{\argmin}{arg\,min}

%%%%%%%%%%%%%%%%%%%%%%%%%%%%%%%%%
\begin{abstract} 
We describe COMET, a block coordinate-descent approach to learn distance metrics from data, using triplet-based ranking loss. The algorithm takes an interior point approach to guarantee that the search remains within the cone of positive definite matrices, and avoid costly projections to the cone. This is achieved by repeatedly optimizing a single row and column, corresponding to one input feature. 
Our approach does not enforce the updates to be positive definite. Instead, it uses the Schur complement condition to guarantee that updates result with a positive definite metric matrix.
We show that the computational complexity of COMET is competitive and derive convergence bounds showing linear convergence with high probability. Our experiments show that COMET outperforms competing methods. It is evaluated on two benchmarks in tasks of retrieving similar images and similar text documents. Additionally,  COMET is naturally set up for learning metrics in face of a growing and changing feature set, because the optimization is performed per-feature.
\end{abstract} 

\section{Introduction}
Learning a measure of pairwise distance among data samples is a fundamental task in machine learning. Learned metrics can be used to retrieve images that are similar to a query image, or recommend a news webpage given a page that a user visits. It can also be used as a representation for supervised learning techniques based on distances, such as nearest-neighbors or kernel methods \cite{kulis2012survey}. 

Many metric learning (ML) approaches focus on learning linear transformations of the data, sometimes called Mahalanobis metrics. The metrics are learned such that the distances among samples in the projected space agree with some given measure of similarity (labels). For given vectors $x$ and $y$ this problem amounts to learning a bi-linear form  $D^2_{\W}(x,y) \eqdef (x-y)^T\W(x-y)$ parametrized by the matrix $\W$. When $\W$ is a positive definite (PD) or positive semidefinite (PSD) matrix, it can be factored as $\W=A^TA$,  and $D^2_{\W}(x,y)$ becomes equivalent to $||Ax-Ay||^2_2$ and is therefore a proper distance function.

Usually, learning positive definite matrices is cast as a convex optimization problem, using the fact that the set of PSD matrices is convex. However, in practice it is still often very challenging to learn the matrices efficiently when presented with high dimensional and massive data. For instance, a naive approach of repeatedly projecting onto the PD cone following gradient steps is prohibitively expensive in run time, as each projection requires an eigendecomposition which is cubic in the number of features. As a result, finding efficient optimization algorithms for metric learning is an ongoing important problem.

An important challenge for metric learning is the case where the set of features is not fixed in advance, but changes with time. This is a typical scenario in many real life applications of learning: as more data accumulates, it is possible to estimate more parameters accurately, so more features and signals are gradually added to existing systems. A common example is that given more text documents, the vocabulary grows. It is therefore desirable to develop algorithms for learning metrics in face of a growing feature set. 

Here we describe COMET, a {\em{CO-ordinate-descent METric learning}} algorithm. It is a computationally efficient block-coordinate optimization method that updates the matrix one row and column at a time, while at the same time efficiently limiting search to the interior of the PD cone without use of eigendecompositions. COMET converges in a geometric rate to the globally optimal value, and is naturally set up for learning metrics in face of a growing feature set. Evaluations on two benchmark datasets show that it performs better than other scalable metric learning methods which avoid costly projections. 

%Optimizing over the PD cone is challenging in the block-coordinate framework, since the PD constraint ties together all of the matrix coordinates. We show how to efficiently overcome this challenge by using the Schur complement to obtain faster and more stable step sizes.


\section{Related work}
There has been intensive work on learning distance metrics and similarity measures from data, see \citet{bellet2013survey, kulis2012survey} for recent surveys. We consider here methods for learning linear Mahalanobis distance matrices as described above. A major challenge in this domain has been to efficiently enforce the similarity matrix to be positive. Learning a PSD matrix efficiently is hard because it requires maintaining all of the matrix eigenvalues non-negative during optimization, a global constraint that does not factorize easily.

The simplest approach to enforcing positivity during optimization is to project the learned matrix onto the cone of PSD matrices. This projection amounts to solving an eigen deconvolution problem and is therefore costly (generally cubic in the feature dimensionality). In some cases, the number of projections can be cleverly cut down but each projection is still slow \cite{qianHD, qian}, and in some problems projections can be computed more efficiently \cite{shalev2004online}. A second common approach has been to learn the parameters of factored model $A^TA$. In this case, the learning problem is no longer convex. 

Another line of works takes an interior-point approach, avoiding projections by limiting optimization within the cone of PSD matrices. \citet{davis2007information} and \citet{lego} used a logbarrier regularizer term. 
Another type of interior points method views a PSD matrix as a combination of other simpler PSD matrices. Very recently, \citet{hdsl} introduced HDSL, showing how to learn a PSD matrix as a weighted combination of rank-1 sparse PSD update matrices, which are all zeros except for $2\times2$ entries correpsonding to pairs of feature. \citet{boost} introduced BoostMetricwhich learns the metric matrix using rank-1 (PSD) updates which are generated by solving a dual optimization problem in a boosting-based process. See also \cite{bi2011adaboost}.

We take here an approach based on minimizing a strongly convex function using block-coordinate descent. There is a well established body of work analyzing the convergence of block-coordinate descent, eg \cite{nesterov2012efficiency,richtarik2014iteration}. We discuss this further in section 5 below.


% ===============================================
\section{The learning setup}
We address the problem of learning a distance metric over a set of
entities, like images or text documents based on data regarding their
relative similarities.

Formally, following \cite{OASIS}, let $\cal{P}$ be a set of entities
$\{\p_1,...,\p_N\}$ each represented as a vector in $\Rd$.  We measure
the similarity of two samples $\q, \p \in \cal{P}$ using a bilinear
form parametrized by a model $\W \in \mathbb{R}^{d \times d}$.
\begin{equation}
  S_{\W}(\q, \p) = \q\T \W \p \quad.
\end{equation}
When the matrix $\W$ is PSD, it can be factored as $\W = A^TA$ and used to define a distance measure over pairs of data points. This is achieved by using the root matrix $A$ to transform the data: Given two data points $x$, $y$, their distance through the matrix $\W$, $(x-y)^T\W(x-y)$, is equivalent to a Euclidean distance in the transformed space  $\sqrt{(Ax-Ay)^T(Ax-Ay)} = ||Ax-Ay||_2$. 

We assume that a weak form of supervision is given in the form of
ranking relation over triplets. This form of supervision is often easy
to obtain and has been shown to achieve good performance
\cite{weinberger2006dml,OASIS,qian}. Specifically, we assume we have access to
triplets of entities from $\cal{P}$, where each triplet $t$ consists of
a ``query'' instance $\qt \in \cal{P}$, and two instance $\pt^{+}, \pt^{-} \in \cal{P}$ such that $\qt$ is more similar to $\pt^{+}$
than to $\pt^{-}$.

We aim to find a similarity measure $S_{\W}$ that agrees with the ranking of these triplets, namely, $S_{\W}(\q, \p^{+}) > S_{\W}(\q,
\p^{-})$. To achieve this, we may use one of the folowing triplet loss functions
\begin{align}
\label{single-triplet-hinge-loss}
l_{\W}^h(\qt, &\pt^{+}, \pt^{-}) = [1-\qt\T\W\pt^+ + \qt\T\W\pt^-]_{+}
 \\ \nonumber
 \label{single-triplet-hinge-loss2}
l_{\W}^{hs}(\qt, &\pt^+, \pt^-) = [1-\qt\T\W\pt^+ + \qt\T\W\pt^-]_{+}^2
 \\ \nonumber
 \label{single-triplet-log-loss} 
l_{\W}^{log}(\qt, &\pt^+, \pt^-) = \\ 
&log\left(1+exp(-\qt\T\W\pt^+ + \qt\T\W\pt^-)\right) \nonumber ,
\end{align}
where $[z]_{+} \eqdef max(0,z)$.  Given a batch of $T$ triplets, and adding a Frobenius regularization term, we therefore wish to solve the following regularized optimization problem
\begin{eqnarray}
  \min_{\W}& \sum_{\trip=1}^T  l_{\W}(\qt, \pt^+, \pt^-) + \frac{\beta}{2} \frobsq{\W}
 \\  \nonumber
   \rm{s.t.}& \W \succ 0 \quad,
  \label{hingelt}
\end{eqnarray}
where $l_{\W}$ is any of the triplet loss functions, $\frob{\W}$ is the Frobenius norm of the matrix $\W$, $\beta$ is the regularization weight and $\W \succ 0$ refers to $\W$ being positive definite. This optimization problem is convex in $\W$ since the objective is a sum of convex functions in $\W$, and the domain $\W \succ 0$ is convex as well.

In several previous metric learning approaches \cite{qianHD, qian}, the constrained optimization problem is solved in an online way by making stochastic mini-batches gradient steps, while repeatedly projecting back to the convex cone of positive definite matrices. This projection amounts to solving an eigendecomposition problem and is therefore costly in run-time.

An alternative approach is to use a log-barrier term to avoid projecting onto the PD cone \cite{davis2007information,lego}. The log-barrier term for the PD cone is $- \log \det (\W)$. This gives us the following optimization problem:
\begin{multline}
\label{eq-logdet-loss}
  L(W) = 
  \min_{\W} \sum_{\trip \in T}  l_{\W}(\qt, \pt^+, \pt^-) \\ 
  - \alpha \log \det(\W) + \frac{\beta}{2} \frobsq{\W}
\end{multline}
where $\alpha$ is a hyper-parameter that determines the weight of the
$\log \det$ barrier term.


The loss in \eqref{eq-logdet-loss} can be minimized using gradient descent (GD), computing the gradient w.r.t. $\W$ for a set of triplets
\begin{multline}
  \frac{\partial {L (\W)}}{\partial \W} = \sum\limits_{t\in T}{\{
  [\tfrac{1}{2}[\q_{t}\Delta\p_{t}\T + \Delta\p_{t}\q_{t}\T]  }
  {l'}\triplet\} \\- \alpha \W^{-1} + \beta \W
%  (\lambda_{W}^t(\W))\} \\- \alpha \W^{-1} + \beta \W
  \label{gradMtx}
\end{multline}
%{l'}(\lambda_{t}^{\W})
%where $l'(x) \eqdef \frac{\partial {l(x)}}{\partial x}$, $\Delta\vec{p}_{t} \eqdef (\vec{p}_{t}^{-} - \vec{p}_{t}^{+})$
where $\Delta\p_t = \p_t^- - \p_t^+$, and $l'(x) \eqdef \frac{d{l(x)}}{dx}$ is the outer derivative of the loss function (see Appendix A for derivation details). For example, for the hinge-loss, 
$l'\triplet$ 
%${l'}(\lambda_{t}^{\W})$ 
is an indicator function which obtains a value of $1$ if the loss is positive, and zero otherwise.
\todo{$\lambda^{t}_{\W}$ should be defined earlier}


% ============================================================
\section{Coordinate-descent metric learning}

The learning setup described above is commonly studied, but 
optimizing it using a gradient approach \eqref{gradMtx} has two main drawbacks: computation efficiency, and a modelling consideration.

First, matrix learning problems typically have a large number of parameters because the model scales quadratically with the feature dimension. %Repeatedly computing the gradient over all features is often wasteful, and approximating the direction based on few features may be faster.
Furthermore, the $\log \det$ regularizer yields  $\W^{-1}$ in the gradient so naive implementations of inverting the matrix are slow.

A second type of problem concerns the effect of the $\log \det$ barrier on the solution space. In many applications the actual matrix minimizing the above loss is near the boundary of the cone. This is the case when using human judgment on similarity as a supervision signal, and can be seen by optimizing without using the PSD constraint - one often finds the optimum outside or near the boundary of the PSD cone.
The problem is that if the only component keeping the solutions inside the PSD cone is the $\log \det$ term, it may distort the gradients near the boundary of the convex set. 

We propose an algorithm that alleviates these problems by using efficient block-coordinate descent that keeps optimization within the boundaries of the PSD cone.
Block-coordinate descent also enjoys provably fast convergence rates, and is especially useful when the block update can be performed efficiently, as we show below.

Our algorithm applies block-coordinate descent approach as follows.
%First, $\W$ is initialized to some value $\W^0$ . Then, a
%  Guys, I think the above is prtty trivial and can be skipped in the non formal description.
At each step, a single feature is drawn uniformly at random, all the matrix entries on its row and columns are treated as a block, and get updated. Importantly, we compute analytically a bound on the size of the update step, which guarantees that the updated matrix remains positive definite.

More formally, we perform the block updates as follows. First, since $\W$ is PD it is also symmetric, and we can replace $\W$ in \eqref{eq-logdet-loss} with $\tfrac{1}{2}(\W + \W\T)$.  This guarantees that the gradient resulting from \eqref{gradMtx} is also symmetric (see details of derivation in Appendix A). We draw a feature $k \in \{1 \ldots d$\} uniformly at random, and define the matrix $G$ to be a matrix that is all zeros except the values of $\grd$ lying in the $i$-th row and $i$-th column. Finally we update the weight matrix using 
\begin{equation}
    \newW = \W^t +\eta G
\label{updateEq}
\end{equation}
where $\eta$ is the step size.
\ignore{
Since PD matrices are symmetric, we require the update $\mat{G}$ of \eqref{updateEq} to be symmetric:
\begin{equation}
  \mat{G} = \vec{u}\cdot\vec{e_k}\T + \vec{e_k}\cdot\vec{u}\T
  \label{gradMtx}
\end{equation}
where $\vec{u}$ is a column vector that equals the column $k$ of the
(symmetric) gradient matrix of the objective \eqref{gradMat},
$\vec{e_k}$ equals an elementary vector for selecting a column $k$ of
a matrix.
}

% ----------------------------------------
\subsection{Selecting the step size $\eta$}\label{subsec:step}
Taking a coordinate step may take $\newW$ out of the PD cone. We now show how to bound the step size to guarantee that $\newW$ remains PD using the Schur complement condition for positive definiteness.

Without loss of generality, assume that the current round updates
the first feature ($k = 1$).  We then write the pre- and post-update
matrices, as
\begin{equation}
  \W = \left[ \begin{matrix} \C & \B\T \\ \B & A \end{matrix} \right],
  \quad
  \newW = \left[ \begin{matrix} \C^* & \B^*\T \\ \B^* & A^* \end{matrix} \right]
  \label{schurNotationPreUpdate}
\end{equation}
 where $\C = \Wscalar \in \R$ (a scalar), $\B = \Wvec \in
\R^{d-1}$ (a column vector) and $A = \W_{2:d,2:d} \in \R^{(d-1)
\times (d-1)}$. Similarly for $A^*$, $\B^*$ and $\C^*$.


According to the Schur complement condition for positive definiteness
\citep[p. 650]{boyd2004convex}, $\newW$ is PD if and only if both
$A^*$ and $\C^* - \B^*\T A^{*-1} \B^*$ are positive definite.
%\begin{equation}
%  \newW \succ  0 \Leftrightarrow (A^* \succ  0, \C^* - \B^*\T A^{*-1} \B^* \succ 0)
%  \label{schurCondPreliminary}
%\end{equation}
Since $W \succ 0$ and $A$ is a minor of $\W$ which is left unchanged by the update, we have $A^* =
A \succ 0$.  Moreover, $\C^* - \B^*\T A^{*-1} \B^*$ is a
scalar, yielding
\begin{equation}
  \newW \succ  0 \Leftrightarrow  \C^* - \B^*\T \invA \B^* >  0
  \label{schurCond}
\end{equation}
%

%
Now let $\uscalar = \C^* - \C$ and $\uvec = \B^* - \B$ be the updated scalar and vector
obtained from $\eta G = \newW - \W$. We expand \eqref{schurCond} and
\eqref{gradMtx} (with $k=1$) yielding a necessary and sufficient condition for $\newW \succ 0$
\begin{equation}
  \begin{array}{ll} 
    (\Wscalar + 2\eta \uscalar) \\
    -(\Wvec + \eta \uvec)\T \invA (\Wvec + \eta \uvec)  & > 0
  \end{array}
  \label{PDUpdateCondNonSimpl}
\end{equation}
Grouping \eqref{PDUpdateCondNonSimpl} as a quadratic inequality in $\eta$, and using the notation from \eqref{schurNotationPreUpdate} we have
%\begin{equation}
%  \begin{array}{ll} 
%    -(\uvec\T \invA \uvec) \, \eta^2 \\
%    +2(\uscalar - \uvec\T \invA \Wvec) \, \eta \\
%    +(\Wscalar - \Wvec \T  \invA \Wvec) & > 0
%  \end{array}
%  \label{PDUpdateCondQuadFormWithW}
%\end{equation}
%or according to \eqref{schurNotationPreUpdate}
\begin{equation}
  \begin{array}{ll} 
    (\uvec\T \invA \uvec) \, \eta^2 \\
    -2(\uscalar - \uvec\T \invA \B) \,\eta \\
    -(\C - \B\T  \invA \B) & < 0 \quad .
  \end{array}
  \label{PDUpdateCondQuadForm}
\end{equation}
For $\eta = 0$ this inequality always
holds since $\W \succ 0$ guarantees that $\C-\B^{\T} \invA \B >0$. As a result,
 \eqref{PDUpdateCondQuadForm} always has a real
root $\eta > 0$. This root provides an upper bound on $\eta$ that guarantees that $\newW$ is PD.  The computational complexity of solving \eqref{PDUpdateCondQuadForm}, assuming $\invA$ is given, is $O(d^2)$ because computing the coefficients involves computing bilinear terms.
Furthermore, $(\newW)^{-1}$ and $\invA$ can be computed efficiently in $O(d^2)$ given $\W^{-1}$ and $\invA$, using the Woodbury inverse matrix identity (see Appendix B).

To conclude, we derived an upper limit for the step size of a block coordinate (row-column) step (\eqref{gradMtx} and \eqref{updateEq}) that guarantees that the updated matrix is PD. The computational complexity for the evaluation
of \eqref{PDUpdateCondQuadForm} is $O(d^2)$, while holding $O(d^2)$ matrix elements in the memory.

\ignore{\todo{Discuss numerical stability ?}}
Our approach is summarized in Algorithm \ref{alg:comet}. We've chosen $\W^0$ to be the identity matrix $I_d$.

\begin{algorithm}[tb]
   \caption{COMET}
   \label{alg:comet}
\begin{algorithmic}[1]
   \STATE {\bfseries input:} training data, max number of steps, $\alpha$, $\beta$
   \STATE {\bfseries initialize:} 
   \STATE Generate a triplet set $T$, Set  $\W  \leftarrow I_d$ , $\W^{-1}  \leftarrow I_d$
   \REPEAT 
   \STATE Select a coordinate $k \in {1..d}$ uniformly at random.
   \STATE Compute $\invA$ using $\W^{-1}$; \eqref{InvA}.
   \STATE Compute the coordinate step gradient $G$; \eqref{gradMtx}.
   \STATE Select the step size $\eta$, with an upper limit from \eqref{PDUpdateCondQuadForm}.
   \STATE Update the metric to $\newW=\W+\eta G$.
   \STATE Update the metric inverse to $\newW^{-1}$; \eqref{InvWwdb}.
   \UNTIL{stopping condition}
\end{algorithmic}
\end{algorithm}

\subsection{Analysis of computational complexity}
We evaluate the computational complexity of a single coordinate step \eqref{gradMat}, including the computation of the gradient and updating of $\W$, $\W^{-1}$ and $\invA$

Consider first the computation of the gradient. For the hinge-loss case $l^{h}_W$, each element $\delta_{i,j}$ of the gradient matrix \eqref{gradMat} equals
\begin{multline}
    \delta_{(i,j)} = \sum\limits_{t\in T}{ [\tfrac{1}{2}[(\vec{q}_{t})_i(\Delta\vec{p}_{t}\T)_j + (\Delta\vec{p}_{t}\T)_i(\vec{q}_{t})_j\T] } \cdot \textbf{1}(\lambda_{W}^t)  \\ 
 - \alpha \cdot \W^{-1}_{i,j} + \beta \cdot \W_{i,j}
\label{gradMatElem}
\end{multline}

For dense data, evaluating the sum over triplets costs $O(|T|)$ operations, where $|T|$ is the number of triplets. However, when the  data is sparse with a sparsity coefficient $\gamma$,  $ 0< \gamma <1 $, then evaluating the sum in \eqref{gradMatElem} costs an average of $O(\gamma^2 |T|)$ operations, because we can accumulate only the elements that are both non-zeros in $(\vec{q}_{t})_i$ and in $(\Delta\vec{p}_{t}\T)_j  $ and likewise for $(\vec{q}_{t})_j$ and $(\Delta\vec{p}_{t}\T)_i$. Computing $\W^{-1}_{i,j}$ and $\W_{i,j}$ costs $O(1)$ since we keep updated versions of $\W$ and $\W^{-1}$. Computing all the gradient elements $\delta_{(k,:)}$ in a single row $k$ then costs $O(d\cdot \gamma^2 |T|)$. Maintaining and updating $\W^{-1}$ and $\invA$, and computing the optimal step size follows from Eqs \ref{InvA}, \ref{PDUpdateCondQuadForm} and \ref{InvWwdb}, each with a cost of $O(d^2)$. The total computational complexity per block-coordinate step is therefore $O(\gamma^2 d |T| + d^2)$.

In order to efficiently evaluate the indicator functions $\{ \textbf{1}(\lambda_{W}^t) \}_{t \in T}$ of \eqref{gradMatElem}. We hold an array of the linear terms $\{\lambda_{W}^t\}_{t \in T}$. Updating a single element $\delta_{i,j}$ of the gradient matrix will force an update of of a sparse number of elements in this array. It also costs $O(\gamma^2 |T|)$ operations, because the elements that change are those that are both non-zeros in $(\vec{q}_{t})_i$ and in $(\Delta\vec{p}_{t}\T)_j$ and likewise for $(\vec{q}_{t})_j$ and $(\Delta\vec{p}_{t}\T)_i$.

Considering the total number of updates 
the overall complexity of COMET is $O(N \cdot (\gamma d)^2 |T| + N \cdot d^3)$ while taking $N \dot d$ coordinate steps, where N is the number of iterations over all the coordinates. We found empirically that COMET converges within $N= 5 - 10$.

As a comparison, consider using SGD or mini-batches for the objective \eqref{hingelt} and projecting onto the PD cone every $m$ triplets ($m << |T|$). As proposed in \citet{OASIS,qian}.
The computational complexity per data pass becomes $O((\gamma d)^2 |T| + \frac{|T|}{m} d^3)$ which is larger compared to COMET. In fact, it only reaches the complexity of COMET when projections are very rare $m \propto |T|$.

\ignore{
As another comparison, consider a simple method like OASIS \cite{OASIS} whose update steps cost $d^2$ and avoids maintaining a PSD matrix, and uses one-time projection to the PSD cone after learning has ended (costing $d^3$). Their runtime complexity is also $O(\gamma d)^2 |T| + d^3)$. 
}


With regard to memory footprint, keeping the data triplets in memory takes $O(\gamma d |T|)$ elements and holding $\W$ and $\W^{-1}$ costs $O(d^2)$. The total memory usage is $O(\gamma d |T| + d^2)$. \ignore{The fact that COMET performs local updates of $\W$ calls for parallelizing it, as done with other coordinate-descent approaches.}%I'm not sure we can parallelize, since the inverse term ties everything together


\section{Convergence rate}
Our method is based on minimizing a strongly convex function using block-coordinate descent. There is a well established body of work showing that with non-overlapping blocks, block-coordinate descent iterates converge w.h.p. in a linear rate to the optimum value \cite{nesterov2012efficiency,richtarik2014iteration}.
However, the blocks we use in our method are overlapping - for example the $(1,2)$ coordinate of the matrix is a part of both the 1\textsuperscript{st} and the 2\textsuperscript{nd} column-row. To address this case, we use a more general convergence result applicable to overlapping blocks, given by \citet{richtarik2013optimal}. \citeauthor{richtarik2013optimal} give a very general result, suitable for \emph{any} distribution over the set of coordinate subsets. 
Specifically relevant in our case, \citeauthor{richtarik2013optimal} give sufficient conditions for linear convergence of overlapping block-coordinate descent for a strongly convex smooth objective. We use a relatively simple distribution over coordinate subsets: we have $d$ overlapping blocks corresponding to the column-rows of the matrix, and we sample them with uniform probability $\frac{1}{d}$.


The step sizes implied by the convergence theory are usually conservative underestimates, especially since many of the constants involved in obtaining the step-sizes cannot be evaluated exactly but can only be upper bounded. In practice, we found that much faster convergence is gained using larger steps while staying within the PD cone, using the procedure described in detail in section \ref{subsec:step}.

To show convergence, we must prove our objective satisfies two assumptions: Assumption 1, called ``Expected Separable Overapproximation'', is that in expectation over the choice of blocks the function is smooth w.r.t. a certain inner product measure induced by the coordinate probabilities. Assumption 2 is that the objective is strongly convex. In addition, for technical reasons the objective must be differentiable. This means that technically our proof is only valid for the squared hinge-loss and log-loss, but not to the non-differentiable hinge-loss.

To fulfill the conditions in \cite{richtarik2013optimal}, we must slightly modify the objective function $L({\W})$. The objective $L(\W)$ is strongly convex but is not smooth, since the gradient of the $\log \det$ term is unbounded near the envelope of the positive definite cone. Let $\tilde{L}({\W}) = L({\W + \kappa I_d})$, where $I_d$ is the $d \times d$ identity matrix, and $\kappa$ is a fixed parameter.
Note that our algorithm can easily minimize $\tilde{L}$, the only difference being that we now need to maintain and update both $\W^{-1}$ and $(\W+\kappa I_d)^{-1}$, which does not change the asymptotic computational complexity. The additional $\kappa I_d$ term acts as a prior, where we add a constant Euclidean distance term to the distance we learn. 

We show that the modified objective $\tilde{L}$ obeys Assumptions $1$ and $2$ of \citet{richtarik2013optimal}. Thereby, according to Theorem 3 of \citeauthor{richtarik2013optimal}, Algorithm \ref{alg:comet} converges to the optimum value in a linear rate.

See Appendix  for the proofs relating to the statements in this section.
\begin{lemma}[Smooth objective]
\label{lem:smooth}
Let $\tL=\sum\limits_{t\in T}{l_{\W + \kappa I}(\vec{q}_t, \vec{p}_{t}^{+}, \vec{p}_{t}^{-})} - \alpha \cdot \log \det(\W + \kappa I) + \tfrac{\beta}{2}  \cdot \| \W + \kappa I \|_{F}^{2}$, where $l_{\W + \kappa I}$ is either the squared hinge loss or the log-loss, and $\tL$ is defined over the positive semidefininte cone. 
Let $\Hh^i \in \R^{d \times d}$, $i=1 \ldots d$, be a symmetric matrix with non-zero entries only on the $i$-th row and column.

For any $\W$ and $\Hh^i$ such that $\W + \Hh^i$ is PSD, there exists a positive constant $M_i$ such that:
\begin{align}
\label{eq:ineq}
&\tilde{L}(\W + \Hh^i) \leq \tL + \langle \grd, \Hh^i \rangle + \frac{M_i}{2} \frobsq{\Hh^i} = \\
&\tL + \sum_{k,l=1}^d  \grdkl \Hh_{kl}^i + \frac{M_i}{2} \sum_{k,l=1}^d  (\Hh_{kl}^i)^2, \nonumber
\end{align}
with the constant $M_i \leq  2 \sum_{t=1}^T (\qt_i^2 +{\Delta\vec{p}_{t}}_i^2) + \frac{\alpha d}{\kappa ^2} + \beta$.

\end{lemma}

Let us define a matrix $\Pp \in \R^{d \times d}$ such that $\Pp_{ij} = \frac{2}{d}$ for $i \ne j$, $\Pp_{ii} = \frac{1}{d}$. $\Pp$ is defined such that $\Pp_{ij}$ is the probability of updating the $(i,j)$ entry of the matrix $\W$ at any given iteration. To show our method converges in a linear rate, we must show that $\tL$, $\Pp$ and the constants $M_i$ satisfy the ``Expected Separable Overapproximation'' assumption presented by \citet{richtarik2013optimal}:

\begin{lemma}[Expected Separable Overapproximation]\label{lem:ESO}
For any symmetric $\Hh \in \R^{d \times d}$ such that $\W + \Hh$ is PSD, let $\Hh^i \in \R^{d \times d}$, $i=1 \ldots d$ be identical to $\Hh$  on the $i$-th row and column, and $0$ elsewhere. Then:
\begin{align*}
&\mathbb{E}_{i \sim uniform 1 \ldots d} \left[ \tilde{L}(\W + \Hh^i) \right] \leq \\
&\tL + \sum_{k,l=1}^d  \grdkl \Hh_{kl} \Pp_{kl} + \frac{1}{2} \sum_{k,l=1}^d   M_k (\Hh_{kl})^2 \Pp_{kl}
\end{align*}
\end{lemma}



\begin{theorem}
Let $\W^t$ be the $t$-th iterate of Algorithm \ref{alg:comet} with objective function $\tL$, sampling each column-row $i$ with uniform probability and using step sizes $\eta_i \leq \frac{1}{M_i}$. Let $\tilde{L}^*$ be the optimal value of $\tL$ on the PSD cone. Let $\beta^* \geq \beta$ be the strong convexity parameter of $\tL$, $M^1 = \max_i M^1_i$, and $\rho >0, \epsilon>0$.

If $t > \frac{d M^1 + \alpha (d/\kappa)^2 + d\beta}{\beta^*} log \left( \frac{\tilde{L}(W^0) - \tilde{L}^*}{\epsilon \rho}\right)$ then: $$Prob(\tilde{L}(\W^k) - \tilde{L}^* \leq \epsilon) \geq 1-\rho.$$
\end{theorem}


\section{Experiments}
We evaluate COMET on two datasets and compare its performance with four metric-learning approaches. All the approaches we compare with learn a Mahalanobis metric matrix and avoid repeated projections to the positive definite cone. 


\begin{figure}[ht]
\vskip 0.2in
\begin{center}
\centerline{\includegraphics[width=\columnwidth]{COMET_convergence}}
\caption{ \textit{Precision-at-top-k} of COMET, evaluated on the test set, as a function of the number of coordinate steps, normalized to the data dimension. Error bars denote the standard error of the mean across 5 random train/test partitions (80\%/20\%) }
\label{cometConvergeFig}
\end{center}
\vskip -0.2in
\end{figure} 



\subsection{Competing approaches}

\textbf{COMET}. The algorithm described in \ref{alg:comet}. We experimented with the linear hinge-loss for the triplets loss, see \eqref{eq-logdet-loss}.

\textbf{Euclidean}. The Euclidean distance in the original feature space. COMET is initialized using the identity matrix, which is equivalent to this distance measure.

\textbf{HDSL} \cite{hdsl}. An approach tuned for high-dimensional sparse data. HDSL learns a convex combination of rank-1 PSD matrices that are all zeros except for a $2\times2$ pair of features elements. It iteratively adds these matrices, one pair of features at a time, to control the number of active features.

\textbf{LEGO} \cite{lego}. This approach uses a $\log \det$ barrier term to enforce matrix PD. The main variant of LEGO aims to fit a given pairwise distances. We used another variant of LEGO that, like COMET, learns from relative distances. Loss is incurred for same-class samples that are farther than a certain distance, and different-class samples closer than certain distance.

\textbf{BoostMetric} \cite{boost}. Based on the observation that any positive semidefinite matrix can be decomposed into linear positive combination of rank-1 matrices, BoostMetric uses rank-1 PSD matrices as weak learners within a boosting based learning process.

% -----------------------------
\subsection{Datasets}
We evaluate COMET on two benchmark datasets.

\textbf{Reuters CV1} is a widely used collection of English text documents. We used the 4-class subset of Reuters CV1 introduced in \cite{CaiRCV14} that was recently tested for metric learning in \cite{hdsl}. We used the \textit{infogain} criterion \cite{infogain} to select a subset of 5000 features that conveyed high information about the identity of the class. This is a discriminative criterion, which measures the number of bits gained for category prediction by knowing the presence or absence of a term in a document. Each document was represented as a bag of words, where the weights of the selected features were normalized using \textit{tf-idf}. The sparsity of this dataset, after selecting the 5000 infogain features, is 1.3\%

We used 100,000 triplets (and 200,000 LEGO constraints) for training on Reuters CV1. This is the same scale that was used in \cite{hdsl}. For training HDSL, we took 8000 iterations as in \cite{hdsl}. Due to the large amount of features, BoostMetric could not converge on this dataset because it was too slow and used a large amount of memory.

\textbf{Caltech256} is a dataset of labeled images used for visual object recognition. We used the subset of 50 classes tested for metric learning in \cite{OASIS}. This set contains 65 images per class (total of 3250 images), represented with ~1000 \textit{bag-of-local-descriptors} features provided by these authors. The sparsity of this dataset is 3.3\%

We used 135,000 triplets (and 300,000 LEGO constraints) for training, roughly as was used in \cite{OASIS}. To select the number of iterations in training HDSL we used early stopping on a validation set. BoostMetric was slow on this dataset, and used a large amount of memory. For a fair comparison, we took the number of COMET coordinate steps to be the maximal number of BoostMetric rank-1 updates.

\subsection{Experimental setup}
In both datasets, two samples are considered similar if they share the same class label. Each data set is tested on a two layer 5 fold cross validation experiment with 80\%/20\% random splits. We use the same (frozen) random splits across all approaches. We trained all learners with the exact same set of triplets, except for LEGO that uses pairs constraints. We verified that we choose the triplets/constraints number in a regime such that test performance converges (figures not shown due to space constraints). We generated triplets randomly while keeping a fixed number of triplets per query sample.

% -----------------------------
\subsection{Evaluation measures}
We evaluated the performance of all algorithms using standard ranking precision measures based on nearest neighbors. For each query instance in the test set, all other test instances were ranked according to their similarity to the query instance. The number of same-class instances
among the top k instances (the k nearest neighbors) was computed. When averaged across test
instances, this yields a measure known as \textit{precision-at-top-k},
providing a precision curve as a function of the rank $k$.

% -----------------------------
\subsection{Results}
 We evaluated the \textit{precision-at-top-k} on the test set, as a function of $k$ neighbours and averaged the results across 5 random train/test partitions (80\%/20\%).
Figure \ref{cometConvergeFig} traces the \textit{precision-at-top-k} over the test sets as it progresses during learning. We observe that convergence is usually achieved after (6 to 8)$\times d$ coordinate steps.
Figure \ref{precFig} compares the precision obtained with COMET, with four competing approaches, as described above. COMET achieved consistently superior or equal results throughout the full range of k (number of neighbours) tested. 

Surprisingly, when studying the optimal values of hyper parameters, we found that the Frobenius regularizer obtained very small weights. Setting its coefficient to zero did not harm the performance.

\begin{figure}[!h]
\vskip 0.2in
\begin{center}
\centerline{\includegraphics[width=\columnwidth]{Precision_at_K_all_datasets}}
\caption{ (best seen in color) Comparison of the performance of COMET, LEGO, BoostMetric, HDSL and the Euclidean metrics across the datasets . Each curve shows the \textit{precision-at-top-k}, evaluated on the test set, as a function of $k$ neighbours. The results are averaged across 5 train/test random partitions (80\%/20\%), error bars are standard error of the means.}
\label{precFig}
\end{center}
\vskip -0.2in
\end{figure} 




\section{Summary}
We presented an approach for learning a distance metric from data samples,  continuously restricting the solutions to the positive cone, but avoiding runtime-costly projections. The approach is based on block-coordinate-descent and iteratively updaes a single row-column in the matrix that correpsonds to a single feature. 




% ==============================================================
\appendix
\section*{Appendix A: Details of gradient derivation}
\label{appendix-grad}

To compute matrix gradient step  $\frac{\partial {l_t (\W)}}{\partial \W}$ of an arbitrary triplet $t$, we denote the linear part of the hinge loss of a triplet $t$ by
%\begin{equation*}
$\lambda_{W}^t \eqdef 
%\lambda_t(\W, \qt, \pt^{+}, \pt^{-}) \eqdef 
1-\qt\T \W \pt^{+} + \qt\T\W\pt^{-}.$
%\end{equation*}
$\W$ is PD and therefore symmetric. We enforce its gradient to be symmetric by replacing $\W$ with $\tfrac{1}{2}(\W + \W\T)$.
The derivative of the ranking loss is then given by
\begin{equation}
\frac{\partial {l_{\W}^{t}}}{\partial \W} = \tfrac{1}{2}[\vec{q}_{t}\Delta\vec{p}_{t}\T  + \Delta\vec{p}_{t}\vec{q}_{t}\T]\cdot {l'}(\lambda_{W}^t)
\label{dlossranking}
\end{equation} where $l'(x) \eqdef \frac{\partial {l(x)}}{\partial x}$, $\Delta\vec{p}_{t} \eqdef (\vec{p}_{t}^{-} - \vec{p}_{t}^{+})$.

When using the hinge loss \eqref{hingelt}, the loss in \eqref{dlossranking} is not differentiable at $\lambda_{W}^t = 0$, but it has a sub gradient matrix
\begin{equation}
\frac{\partial {l_t (\W)}}{\partial \W} = \tfrac{1}{2}[\vec{q}_{t}\Delta\vec{p}_{t}\T  + \Delta\vec{p}_{t}\vec{q}_{t}\T] \cdot \textbf{1}(\lambda_{W}^t),
\end{equation}
where $\textbf{1}(x)$ is an indicator for $x>0$. This gradient is non-zero when the hinge loss \eqref{hingelt} is positive and vanishes otherwise. Replacing the hinge loss by a logarithmic loss yields a continuous $\operatorname{sigmoid}(\lambda_{W}^t)$ instead of $\textbf{1}(\lambda_{W}^t)$.  
%We emphasis that 
%$\frac{\partial {l_t (\W)}}{\partial \W}$ is a matrix, which %represents the gradient of $l_{\W}^{t}$ with respect to each %the elements of $W$.
The matrix gradient of $\tfrac{1}{2} \| \W \|_{F}^{2}$ equals $\W$ and the matrix gradient of  $\log \det(\W)$ equals $\W^{-1}$, yielding 
%Therefore, the matrix gradient of the objective function 
% $L(\W)$ can be easily determined: 
\begin{multline}
\frac{\partial {L (\W)}}{\partial \W} = 
\sum\limits_{t\in T}{\{ [\tfrac{1}{2}[\vec{q}_{t}\Delta\vec{p}_{t}\T + \Delta\vec{p}_{t}\vec{q}_{t}\T \} } \cdot \textbf{1}(\lambda_{W}^t)] \\- \alpha \cdot \W^{-1} + \beta \cdot \W 
\label{gradMat}
\end{multline}


\section*{Appendix B: Updating the inverse matrices}
\label{appendix-inverse}

To compute efficienly  the updates of \eqref{PDUpdateCondQuadForm} we update $\W^{-1}$ following a block coordinate step, and derive $\invA$ from $\W^{-1}$ before the next step. Both terms take $O(d^2)$ to compute.

$\W^{*-1}$ can be easily computed using the Woodbury matrix
identity \cite{woodbury1950inverting}. We rewrite  \eqref{updateEq} and \eqref{gradMtx} using $\newW = \W + \eta G = \W+\mat{\widetilde{G}}$
  %\label{updateEqWDB}
and write
\begin{equation}
  \mat{\widetilde{G}} = \mat{U}\mat{C}\mat{V} = \left[ \begin{matrix}
      \vec{u} & \vec{e_k} \end{matrix} \right] \left[ \begin{matrix}
      \eta & 0 \\ 0 & \eta \end{matrix} \right] \left[ \begin{matrix}
      \vec{e_k}\T \\ \vec{u}\T \end{matrix} \right],
  \label{gradMtxWDB}
\end{equation}
where $\vec{u}$ is a column vector that equals the column $k$ of the gradient matrix of the objective \eqref{gradMat},
$\vec{e_k}$ equals an elementary vector for selecting a column $k$ of
a matrix. 
Using the Woodbury matrix identity gives 
\begin{equation}
    \begin{array}{lcl}
    \W^{*-1} = 
    \W^{-1} - \W^{-1} \mat{U} (\eta^{-1} I_2 + \mat{V}     \W^{-1} \mat{U})^{-1} \mat{V} \W^{-1}
    \end{array}
    \label{InvWwdb}
\end{equation}

Last, we evaluate $\invA$ before a coordinate step given $\W$
and $\W^{-1}$, using the Schur complement and its corresponding
notation \eqref{schurNotationPreUpdate}:
\begin{equation}
\begin{array}{l}
 \W^{-1} \!\!=\!\! 
 \left[ \begin{array}{cc} s & -s \B\T \invA \\ -s \B\T \invA\T &  \invA \!+ \!\invA \B s \B\T \invA  \end{array}  \right]
\end{array}
\label{BlockInvW}
\end{equation}
where $s= \C-\B\T \invA \B$ is a scalar denoting the Schur Complement. This gives us four terms: (1) $s = \W^{-1}_{1,1}$, (2)
$  -s \B\T \invA = -\W^{-1}_{1,1} \B\T \invA = \W^{-1}_{1,2:d}$, (3) $\B\T \invA = -\frac{\W^{-1}_{1,2:d}}{\W^{-1}_{1,1} }$ and (4) $\invA\B = (\B\T \invA)\T$. Subtituting them in the lower right block of \eqref{BlockInvW} yields $\invA + \frac{1}{\W^{-1}_{1,1} } \W^{-1}_{2:d,1} (\W^{-1}_{2:d,1})\T = \W^{-1}_{2:d,2:d}$. Rearranging the last term gives
\begin{equation}
  \invA = \W^{-1}_{2:d,2:d}- \frac{\W^{-1}_{2:d,1} \W^{-1}_{1,2:d}}{\W^{-1}_{1,1}}
  \label{InvA}
\end{equation}
and the computational complexity of computing \eqref{InvA} is $O(d^2)$.


\section*{Appendix C: Proofs}
\label{appendix-proofs}

\ignore{
\begin{applemma}[Smooth objective]
\label{applem:smooth}
Let $\tL=\sum\limits_{t\in T}{l_{\W + \kappa I}(\vec{q}_t, \vec{p}_{t}^{+}, \vec{p}_{t}^{-})} - \alpha \cdot \log \det(\W + \kappa I) + \tfrac{\beta}{2}  \cdot \| \W + \kappa I \|_{F}^{2}$, where $l_{\W + \kappa I}$ is either the squared hinge loss or the log-loss, and $\tL$ is defined over the positive semidefininte cone. 
Let $\Hh^i \in \R^{d \times d}$, $i=1 \ldots d$, be a symmetric matrix with non-zero entries only on the $i$-th row and column.

For any $\W$ and $\Hh^i$ such that $\W + \Hh^i$ is PSD, there exists a positive constant $M_i$ such that:
\begin{align}
\label{eq:ineq}
&\tilde{L}(\W + \Hh^i) \leq \tL + \langle \grd, \Hh^i \rangle + \frac{M_i}{2} \frobsq{\Hh^i} = \\
&\tL + \sum_{k,l=1}^d  \grdkl \Hh_{kl}^i + \frac{M_i}{2} \sum_{k,l=1}^d  (\Hh_{kl}^i)^2, \nonumber
\end{align}
with the constant $M_i \leq  2 \sum_{t=1}^T (\qt_i^2 +{\Delta\vec{p}_{t}}_i^2) + \frac{\alpha d}{\kappa ^2} + \beta$.

\end{applemma}}

\begin{proof}[\bf{Proof of Lemma 1}]

The objective $\tL$ is comprised of three terms - (1) the sum of loss terms, (2) the $\log \det$ term, and (3) the Frobenius regularization term. We will bound each of the separately, denoting the positive bounding constants $M^1_i$, $M^2_i$ and $M^3_i$, respectively.  
%The Frobenius norm term ensures that $\tL$ is at least $\beta$ strongly-convex.

Assuming the instances $\qt$ and $\pt$ are unit normalized, straightforward computation shows that for the term (1), inequality \ref{eq:ineq} holds true for $M^1_i \leq 2 \sum_{t=1}^T (\qt_i^2 +{\Delta\vec{p}_{t}}_i^2)$. %This means that if the features are more or less equally weighted, $M^1_i$ is very roughly on the order of $\frac{T}{d}$.

To show that \ref{eq:ineq} is true for the $- \log \det$ term, we bound the maximal eigenvalue of its Hessian $\mathcal{H}$. 
The Hessian is a $d^2 \times d^2$ PSD matrix, due to convexity and twice-differentiability of $- \log \det$. The Hessian itself is cumbersome to compute.
However, at every point $\mat{X} = \W + \kappa I$, $\W \succ 0$, the Hessian $\mathcal{H}(\mat{X})$ defines a bilinear form $\mathcal{B}_{\mat{X}}\left(\mat{P},\mat{Q}\right)$ on the set of symmetric $d \times d$ matrices. This bilinear form is \citep[Appendix A.4.3]{boyd2004convex}: $$\mathcal{B}_{\mat{X}}\left(\mat{P},\mat{Q}\right) = tr\left(\mat{X}^{-1}\mat{P} \mat{X}^{-1}\mat{Q}\right).$$  We have:
\begin{align*}
&\max eig(\mathcal{H}) = \max_{\|\mat{P}\|_F=1} \mathcal{B}_{\mat{X}}\left(\mat{P},\mat{P}\right) = \\
&\max_{\|\mat{P}\|_F=1} tr\left(\mat{X}^{-1}\mat{P} \mat{X}^{-1}\mat{P}\right) \leq \\
&\max_{\|\mat{P}\|_F=1} \|\mat{X}^{-1} \mat{P}\|_F^2 \leq \|\mat{X}^{-1}\|_F^2 \leq  \\
& d \|\mat{X}^{-1}\|^2 = \frac{d}{\|\mat{X}\|^2} \leq \frac{d}{\kappa^2},
\end{align*}
where in the last line we denote the spectral norm (maximum singular value) of $\mat{X}$ by $\|\mat{X}\|$. The last inequality is due to the fact that $\mat{X} = \W + \kappa I$, $\W \succ 0$.
We therefore have a bound $M^2_i \leq \frac{\alpha d}{\kappa^2}$.

Finally, the constant $M^3_{i}$ for the Frobenius regularization is immediately seen to be $\beta$.

Collecting all the terms together, we obtain an overall bound on the constant: $M_i \leq M^1_{i} + M^2_{i} + M^3_{i} \leq  M^1_{i} + \frac{\alpha d}{\kappa ^2} + \beta$.

\end{proof}

\ignore{
\begin{applemma}[Expected Separable Overapproximation]\label{applem:ESO}
For any symmetric $\Hh \in \R^{d \times d}$ such that $\W + \Hh$ is PSD, let $\Hh^i \in \R^{d \times d}$, $i=1 \ldots d$ be identical to $\Hh$  on the $i$-th row and column, and $0$ elsewhere. Then:
\begin{align*}
&\mathbb{E}_{i \sim uniform 1 \ldots d} \left[ \tilde{L}(\W + \Hh^i) \right] \leq \\
&\tL + \sum_{k,l=1}^d  \grdkl \Hh_{kl} \Pp_{kl} + \frac{1}{2} \sum_{k,l=1}^d   M_k (\Hh_{kl})^2 \Pp_{kl}
\end{align*}
\end{applemma}}

\begin{proof}[\bf{Proof of Lemma 2}]
\begin{align*}
&\mathbb{E}_{i \sim uniform 1 \ldots d} \left[ \tilde{L}(\W + \Hh^i) \right] = \frac{1}{d} \sum_{i=1}^d \tilde{L}(\W + \Hh^i) \stackrel{(a)}{\leq} \\
& \frac{1}{d} \sum_{i=1}^d \left(\tL + \sum_{k,l=1}^d \grdkl \Hh_{kl}^i + \frac{M_i}{2} \sum_{k,l=1}^d  (\Hh_{kl}^i)^2 \right) \stackrel{(b)}{=} \\
& \tL + \sum_{k,l=1}^d \grdkl \Hh_{kl} \Pp_{kl} + \frac{1}{2} \sum_{k,l=1}^d M_k (\Hh_{kl})^2 \Pp_{kl}.
\end{align*}
Inequality (a) is due to Lemma \ref{lem:smooth}. Equality (b) is by a simple counting argument, using the fact the $\Hh^i$ is the restriction of $\Hh$ to the its $i$-th row and column. Note that each off-diagonal element $\Hh_{kl}$ appears twice in the sum over $i$: once when $i=k$ and once when $i=l$. This is accounted for by the elements $\Pp_{kl}$.
\end{proof}


\begin{apptheorem}
Let $\W^t$ be the $t$-th iterate of Algorithm \ref{alg:comet} with objective function $\tL$, sampling each column-row $i$ with uniform probability and using step sizes $\eta_i \leq \frac{1}{M_i}$. Let $\tilde{L}^*$ be the optimal value of $\tL$ on the PSD cone. Let $\beta^* \geq \beta$ be the strong convexity parameter of $\tL$, $M^1 = \max_i M^1_i$, and $\rho >0, \epsilon>0$.

If $t > \frac{d M^1 + \alpha (d/\kappa)^2 + d\beta}{\beta^*} log \left( \frac{\tilde{L}(W^0) - \tilde{L}^*}{\epsilon \rho}\right)$ then: $$Prob(\tilde{L}(\W^k) - \tilde{L}^* \leq \epsilon) \geq 1-\rho.$$
\end{apptheorem}
\begin{proof}
We show that Algorithm \ref{alg:comet} with objective function $\tL$\footnote{With squared-hinge loss or log loss.}, sampling each column-row $i$ with uniform probability, and using step sizes $\eta_i \leq \frac{1}{M_i}$, follows Assumption 1 and Assumption 2 of \citet{richtarik2013optimal}. From this the convergence result follows from \citet[Theorem 3]{richtarik2013optimal} , plugging in our bounds regarding the smoothness and strong convexity of $\tL$.

We first note that our algorithm is indeed a special case of the algorithm presented in \citet{richtarik2013optimal}. Specifically, our algorithm assigns probability $1/d$ to each of the $d$ column-rows of a matrix, and probability $0$ to every other possible choice of coordinates. We update along this block, and the $\log \det$ term acts as a barrier function
assuring us we will stay within the PD cone.

Lemma \ref{lem:ESO} shows that our objective is smooth and satisfies Assumption 1 of \citet{richtarik2013optimal}. Assumption 2 of \citet{richtarik2013optimal} is immediately satisfied because of the Frobenius regularization term, ensuring a strong convexity term  $\beta^* \geq \beta > 0$. The results follows by considering that the probability $\Pp_{ij}$ of updating coordinate $(i,j)$ obeys $\Pp_{ij} \geq 1/d$ and the values of $M_i$ given in Lemma \ref{lem:smooth}.

\end{proof}

\bibliography{comet}
\bibliographystyle{icml2015}

\end{document} 

