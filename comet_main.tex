%%%%%%%%%%%%%%%%%%%%%%%%%%%%%%%%%%%%%%%%%%%%%%%%%%%%%%%%%%%%%%%%%%
%%%%%%%% COMET ICML 2015 
%%%%%%%%%%%%%%%%%%%%%%%%%%%%%%%%%%%%%%%%%%%%%%%%%%%%%%%%%%%%%%%%%%

% Use the following line _only_ if you're still using LaTeX 2.09.
%\documentstyle[icml2015,epsf,natbib]{article}
% If you rely on Latex2e packages, like most moden people use this:
\documentclass{article}

%% COMET packages
\usepackage{amssymb}
\usepackage{amsmath}
\usepackage{mathtools}

%%%%%%%%%%%%%%%%%%%%%%%%


% use Times
\usepackage{times}
% For figures
\usepackage{graphicx} % more modern
%\usepackage{epsfig} % less modern
\usepackage{subfigure} 

% For theorems and proofs
\usepackage{amsthm}

% For citations
\usepackage{natbib}

% For algorithms
\usepackage{algorithm}
\usepackage{algorithmic}

% As of 2011, we use the hyperref package to produce hyperlinks in the
% resulting PDF.  If this breaks your system, please commend out the
% following usepackage line and replace \usepackage{icml2015} with
% \usepackage[nohyperref]{icml2015} above.
\usepackage{hyperref}

% Packages hyperref and algorithmic misbehave sometimes.  We can fix
% this with the following command.
\newcommand{\theHalgorithm}{\arabic{algorithm}}

% Employ the following version of the ``usepackage'' statement for
% submitting the draft version of the paper for review.  This will set
% the note in the first column to ``Under review.  Do not distribute.''
\usepackage{icml2015} 

% Employ this version of the ``usepackage'' statement after the paper has
% been accepted, when creating the final version.  This will set the
% note in the first column to ``Proceedings of the...''
%\usepackage[accepted]{icml2015}


% The \icmltitle you define below is probably too long as a header.
% Therefore, a short form for the running title is supplied here:
%\icmltitlerunning{Metric learning using Block Coordinate Descent}

\begin{document} 

\twocolumn[
\icmltitle{Metric Learning using Block Coordinate Descent}

% It is OKAY to include author information, even for blind
% submissions: the style file will automatically remove it for you
% unless you've provided the [accepted] option to the icml2015
% package.
\icmlauthor{Your Name}{email@yourdomain.edu}
\icmladdress{Your Fantastic Institute,
            314159 Pi St., Palo Alto, CA 94306 USA}
\icmlauthor{Your CoAuthor's Name}{email@coauthordomain.edu}
\icmladdress{Their Fantastic Institute,
            27182 Exp St., Toronto, ON M6H 2T1 CANADA}

% You may provide any keywords that you 
% find helpful for describing your paper; these are used to populate 
% the "keywords" metadata in the PDF but will not be shown in the document
\icmlkeywords{metric learning, similarity learning}

\vskip 0.3in
]

%%%% COMET commands %%%%
\newcommand\todo[1]{\textbf{<ToDo:#1}!}
%\newcommand\mat[1]{\mathcal{#1}}
%\newcommand\mat[1]{\boldmath{#1}}
\newcommand\mat[1]{{#1}}
\renewcommand\vec[1]{\mathbf{#1}}
\newcommand{\T}{{}^\mathsf{T}}
\newcommand{\W}{\mat{W}}
\newcommand{\Z}{\mat{Z}}
\newcommand{\newW}{{\mat{W^*}}}
\newcommand{\eqdef}{\doteq}
\newcommand{\Rd}{\mathbb{R}^d}


\newtheorem{theorem}{Theorem}
\newtheorem{lemma}{Lemma}
\newtheorem{corollary}{Corollary}
\newtheorem{definition}{Definition}


%\DeclareMathOperator*{\argmin}{arg\,min}
%%%%%%%%%%%%%%%%%%%%%%%%%%%%%%%%%

\begin{abstract} 
TBD
\end{abstract} 

\section{Introduction}
Learning a measure of pair-wise similarity or distance among data samples is a fundamental task in machine learning. Learned metrics can be used to retrieve a similar image or document given an example \cite{}, or be used as a representation for a supervised learning technique that is based on distances, like nearest-neighbors or kernel methods \cite{}.


\section{Related work}
There has been a lot of work on learning similarity measures and distance metrics from data, see \cite{} for a review. Specifically, several approaches have been proposed for learning a positive definite Mahalanobis matrix. 

\section{Coordinate-descent metric learning}
We learn the metric using pairwise relations. One approach is to measure the similarity of two samples $\vec{q}, \vec{p} \in \Rd$ using a bilinear form parametrized by a model $\W \in \mathbb{R}^{d \times d}$.
\begin{equation}
S_{\W}(\vec{q}, \vec{p}) = \vec{q}\T \W \vec{p}
\end{equation}
To train our model, we generate a batch of triplets, such that three instances are sampled per triplet ($t$): a query $\vec{q}_t \in \Rd$ and two samples $\vec{p}_{t}^{+}, \vec{p}_{t}^{-} \in \Rd$. We wish that the model assigns a higher similarity score to the pair $(\vec{q}_t, \vec{p}_{t}^{+})$ than the pair $(\vec{q}_t, \vec{p}_{t}^{-})$, and hence use the ranking loss per triplet $t$. We could choose any of linear hinge loss, quadratic hinge loss or logarithmic loss. We've chosen the linear hidge loss which is easy to implement and has demonstrated good results. The linear hinge loss is defined as:
\begin{equation}
l_{\W}(\vec{q}_t, \vec{p}_{t}^{+}, \vec{p}_{t}^{-})=[1-\vec{q}_{t}\T \W \vec{p}_{t}^{+}+\vec{q}_{t}\T\W\vec{p}_{t}^{-}]_{+}
\label{hingelt}
\end{equation}
where $[z]_{+} \eqdef max(0,z)$. Such models have shown to achieve high precision \cite{oasis, qian}.

An outline of our next steps is that we will define an objective function of the total hinge loss over all the triplets, a matrix Frobenius norm regularization and a $\log \det$ regularization over $\W$ that assigns a high cost when $\W$ reaches close to the PD cone boundaries. 
We will take a block coordinate descent approach which ensures that each step is taken within the PD cone. Each coordinate step is taken over a single row and column $k$ of $\W$. An upper bound on the step size is  evaluated using the Schur Complement condition for positive definiteness \todo{Add citation} and we take a step with a step size within that boundary. 

\subsection{The objective}
We define an objective function
\begin{equation}
L(\W)=\sum\limits_{t\in T}{l_{\W}(\vec{q}_t, \vec{p}_{t}^{+}, \vec{p}_{t}^{-})} - \alpha \cdot \log \det(\W) + \tfrac{1}{2} \beta \cdot \| \W \|_{F}^{2} 
\end{equation}
where $T$ denotes the batch of triplets, $\alpha, \beta$ are hyper parameters,  $\log \det(\W) \eqdef \log (\det(\W))$ and $\| \W \|_{F}^{2}$ is the Frobenius norm of $\W$. It is comprised of the total hinge loss over all the triplets and a $\log \det$ regularization over $\W$ that assigns a finite high cost when $\W$ is PD and is close to the PD cone boundaries. 
Our objective is to find $\W$ that minimizes $L(\W)$, such that $\W$ is PD ($\W\succ 0$).

We observe that the PD cone is a convex set and that the objective function is a linear sum of convex functions over the PD cone. Hence the optimization over the objective is a convex problem with a single minima which is global.

\subsection{Gradient evaluation}
Next, we evaluate a matrix gradient step  $\frac{\partial {l_t (\W)}}{\partial \W}$ of an arbitrary triplet $t$. We define the linear part of the hinge loss of a triplet $t$ by
\begin{equation}
\lambda_t \eqdef \lambda_t(\W, \vec{q}_t, \vec{p}_{t}^{+}, \vec{p}_{t}^{-}) \eqdef 1-\vec{q}_{t}\T \W \vec{p}_{t}^{+}+\vec{q}_{t}\T\W\vec{p}_{t}^{-}
\end{equation}

We note that $\W$ is PD and therefore symmetric, therefore we will require its gradient to be symmetric by replacing $\W$ with $\tfrac{1}{2}(\W + \W\T)$.

The derivative of the ranking loss is then given by
\begin{equation}
\frac{\partial {l_{\W}^{t}}}{\partial \W} = \tfrac{1}{2}[\vec{q}_{t}\Delta\vec{p}_{t}\T  + \Delta\vec{p}_{t}\vec{q}_{t}\T]\cdot {l'}(\lambda_t)
\label{dlossranking}
\end{equation} where $l'(x) \eqdef \frac{\partial {l(x)}}{\partial x}$, $\Delta\vec{p}_{t} \eqdef (\vec{p}_{t}^{-} - \vec{p}_{t}^{+})$

Replacing the linear hinge loss \eqref{hingelt} in \eqref{dlossranking}, we observe that it is not differentiable in $\lambda_t = 0$, but it has a sub gradient matrix

\begin{equation}
\frac{\partial {l_t (\W)}}{\partial \W} = \tfrac{1}{2}[\vec{q}_{t}\Delta\vec{p}_{t}\T  + \Delta\vec{p}_{t}\vec{q}_{t}\T] \cdot \textbf{1}(\lambda_t)
\end{equation}
where $\textbf{1}(x)$ is an indicator for $x>0$. Hence it is non-zero when \eqref{hingelt} is positive or zero otherwise. Note that replacing the linear hinge loss by a logarithmic loss would have yield a continues $\operatorname{sigmoid}(\lambda_t)$ instead of $\textbf{1}(\lambda_t)$.  We emphasis that 
$\frac{\partial {l_t (\W)}}{\partial \W}$ is a matrix, which represents the gradient of $l_{\W}^{t}$ with respect to each the elements of $W$.

The matrix gradient of $\tfrac{1}{2} \| \W \|_{F}^{2}$ equals $\W$ and the matrix gradient of  $\log \det(\W)$ equals $\W^{-1}$. Therefore, the matrix gradient of the objective function $L(\W)$ can be easily determined: 
\begin{multline}
\frac{\partial {L (\W)}}{\partial \W} = 
\sum\limits_{t\in T}{\{ [\tfrac{1}{2}[\vec{q}_{t}\Delta\vec{p}_{t}\T + \Delta\vec{p}_{t}\vec{q}_{t}\T \} } \cdot \textbf{1}(\lambda_t)] \\- \alpha \cdot \W^{-1} + \beta \cdot \W 
\label{gradMat}
\end{multline}
\todo{describe computational complexity}
Taking the gradient of the objective which is the sum of of hinge losses over all the triplets in a batch is a straight-forward gradient descent approach and has already been utilized by \cite{qian}. Its drawback is that it can't ensure that the update following the gradient step will keep $\W$ as positive definite. On \cite{qian} a projection to the PSD cone is applied following each update. 

Next, we introduce our novelty for taking a (block coordinate) gradient step that ensures that if $\W$ before the update is PD, then $\newW$ following the update will be kept as PD. 

\subsection{Block coordinate-descent steps}
He we take a (block coordinate) gradient step that ensures that if $\W$ before the update is PD, then $\newW$ following the update will be kept as PD. Each block coordinate step is taken over a single row and column $k$ of $\W$. An upper bound on the step size is evaluated using the Schur Complement condition for positive definiteness and we take a step with a step size within that boundary. 

We define the update:
\begin{equation}
   \newW = \W +\eta\cdot\mat{G}
   \label{updateEq}
\end{equation}
Where $\mat{G}$ is the update matrix based on a single coordinate $k$, $\eta$ is the step size, $\W$ is the pre-update metric matrix, $\newW$ is the post-update metric matrix. We will discuss next on how to choose the steps size $\eta$ and generate $\mat{G}$ row-column $k$ update such that $\newW$ will result as PD. 


Since PD matrices are symmetric, we require the update $\mat{G}$ of \eqref{updateEq} to be symmetric:
\begin{equation}
\mat{G} = \vec{u}\cdot\vec{e_k}\T + \vec{e_k}\cdot\vec{u}\T
\label{gradMtx}
\end{equation}
where  $\vec{u}$ is a column vector that equals the column $k$ of the (symmetric) gradient matrix of the objective, $\vec{e_k}$ equals an elementary vector for selecting a column $k$ of a matrix. 

\subsubsection{choosing the step size $\eta$}
Following, using the Schur complement condition for positive definiteness \todo{AddREF!}, we derive an upper bound for the step size. We will show that taking a step size on $\mat{G}$ which is smaller than the upper bound guarantees that the update \eqref{updateEq} will keep $\newW$ inside the PD cone. 

The pre-update matrix $\W$ can be formulated according to the Schur complement notation as: 
\begin{equation}
\W= \left[ \begin{matrix} C & \vec{B\T} \\ \vec{B} & \mat{A} \end{matrix} \right]
\label{schurNotationPreUpdate}
\end{equation}
where $C = \W_{(1,1)}$ (scalar), $\vec{B} = \W_{(2:,1)}$, $\mat{A} = \W_{(2:,2:)}$

We apply the same notation to $\newW$
\begin{equation}
\mat{\newW}= \left[ \begin{matrix} C^* & \vec{B^*}\T \\ \vec{B^*} & \mat{A^*} \end{matrix} \right]
\end{equation}
According to the Schur complement condition for positive definiteness \todo{AddREF!}, $\newW$ is PD iff $\mat{A}^*$ and $C^* - \vec{B}^*\T \mat{A}^{*-1} \vec{B}^*$ are both positive definite:
\begin{equation}
\newW \succ  0 \Leftrightarrow (\mat{A}^* \succ  0, C^* - \vec{B}^*\T \mat{A}^{*-1} \vec{B}^* \succ  0)
\label{schurCondPreliminary}
\end{equation}
We note that $\mat{A}^* = \mat{A}$ because we only update in this step the row-column $k=1$. $\mat{A}$ is a minor of $\W \succ  0$. Resulting with $\mat{A}^*=\mat{A} \succ  0$. Moreover, $C^* - B^*\T A^{*-1} B^*$ is a scalar. Therefore \eqref{schurCondPreliminary} reduces to:
\begin{equation}
\mat{W^*} \succ  0 \Leftrightarrow (C^* - B^*\T A^{-1} B^* >  0)
\label{schurCond}
\end{equation}
Next, we expand \eqref{schurCond} according to \eqref{updateEq} and \eqref{gradMtx} (with $k=1$) yielding a condition for $\newW \succ  0$
\begin{equation}
\begin{array}{ll} 
(\W_{(1,1)} + 2\eta \vec{u}_{(1)})  \\
-(\W_{(2:,1)} + \eta \vec{u}_{(2:)})\T \mat{A}^{-1} (\W_{(2:,1)} + \eta \vec{u}_{(2:)})  & > 0
\end{array}
\label{PDUpdateCondNonSimpl}
\end{equation}
Grouping \eqref{PDUpdateCondNonSimpl} as a quadratic inequality, we get:
\begin{equation}
\begin{array}{ll} 
-(\vec{u}_{(2:)}\T \mat{A}^{-1} \vec{u}_{(2:)})\eta^2 \\
+2(\vec{u}_{(1)} - \vec{u}_{(2:)}\T \mat{A}^{-1}\W_{(2:,1)})\eta\\
+(\W_{(1,1)} - \W_{(2:,1)}\T  \mat{A}^{-1} \W_{(2:,1)}) & > 0
\end{array}
\label{PDUpdateCondQuadFormWithW}
\end{equation}
or according to \eqref{schurNotationPreUpdate}:
\begin{equation}
\begin{array}{ll} 
(\vec{u}_{(2:)}\T \mat{A}^{-1} \vec{u}_{(2:)}) \eta^2 \\
-2(\vec{u}_{(1)} - \vec{u}_{(2:)}\T \mat{A}^{-1}\vec{B})\eta \\
-(C - \vec{B}\T  \mat{A}^{-1} \vec{B}) & < 0
\end{array}
\label{PDUpdateCondQuadForm}
\end{equation}
As a sanity check, note that the condition \eqref{PDUpdateCondQuadForm} always holds for $\eta = 0$ because $\W\succ0$. Solving \eqref{PDUpdateCondQuadForm} according to $\eta$ results with an upper bound for $\eta$.  The computational complexity of \eqref{PDUpdateCondQuadForm} is $o(d^2)$ because it is a summation of quadratic forms. 

Summing it up, we derived an upper bound \eqref{PDUpdateCondQuadForm} for the step size of a block coordinate (row-column) step (\eqref{gradMtx} and \eqref{updateEq}) that ensures that the update is PD. The computational complexity for the evaluation of \eqref{PDUpdateCondQuadForm} is $o(d^2)$. Nevertheless, evaluating \eqref{PDUpdateCondQuadForm} requires the evaluation of $\W^{-1}$ and $\mat{A}^{-1}$. The most efficient implementation is to update $\W^{-1}$ following a block coordinate step, and to derive $\mat{A}^{-1}$ from $\W^{-1}$ before taking the next step. 

\subsubsection{Complete the step update}

Next, we demonstrate how to evaluate $\W^{*-1}$ following a coordinate step \eqref{updateEq} and how to evaluate $\mat{A}^{*-1}$ before taking the next coordinate step. Both evaluations will cost a computational complexity of $o(d^2)$.

Evaluating $\W^{*-1}$ can be easily taken using the Sbury matrix identity \todo{addref}. By reformulating \eqref{updateEq} and \eqref{gradMtx} according to the Woodbury matrix identity notation, we can write
\begin{equation}
\mat{W^*} = \W+\mat{\widetilde{G}}
\label{updateEqWDB}
\end{equation}
where
\begin{equation}
\mat{\widetilde{G}} = \mat{U}\mat{C}\mat{V} = \left[ \begin{matrix} \vec{u} & \vec{e_k} \end{matrix} \right] \left[ \begin{matrix} \eta & 0 \\ 0 & \eta \end{matrix} \right] \left[ \begin{matrix} \vec{e_k}\T \\ \vec{u}\T \end{matrix} \right]
\label{gradMtxWDB}
\end{equation}

Utilizing the Woodbury matrix identity result with
\begin{equation}
\begin{array}{lcl}
\W^{*-1} = \\
\W^{-1} - \W^{-1} \mat{U} (\eta^{-1} I_{2x2} + \mat{V} \W^{-1} \mat{U})^{-1} \mat{V} \W^{-1}
\end{array}
\label{InvWwdb}
\end{equation}


Last, we evaluate $\mat{A}^{-1}$ before a coordinate step given $\W$ and $\W^{-1}$, using the Schur complement and its corresponding notation \eqref{schurNotationPreUpdate}:
\begin{equation}
\begin{array}{l}
 \W^{-1} =  \\
 \left[ \begin{array}{cc} s & -s \vec{B}\T \mat{A}^{-1} \\ -(s \vec{B}\T \mat{A}^{-1} )\T & (\mat{A}^{-1} + \mat{A}^{-1} \vec{B} s \vec{B}\T \mat{A}^{-1} ) \end{array}  \right]
\end{array}
\label{BlockInvW}
\end{equation}
where $s$ is a scalar that denotes the Schur Complement:  $s \eqdef (\mat{C}-\vec{B}\T \mat{A}^{-1} \vec{B})$. Hence
\begin{equation}
s = \W^{-1}_{(1,1)}
\label{invWkk}
\end{equation}
\begin{equation}
-s \vec{B}\T \mat{A}^{-1} = -\W^{-1}_{(1,1)} \vec{B}\T \mat{A}^{-1} = \W^{-1}_{(1,2:)}
\end{equation}
\begin{equation}
\vec{B}\T \mat{A}^{-1} = -\frac{\W^{-1}_{(1,2:)}}{\W^{-1}_{(1,1)} }
\end{equation}
\begin{equation}
\mat{A}^{-1}\vec{B} = (\vec{B}\T \mat{A}^{-1})\T
\label{invA_Btrans}
\end{equation}
Assign above at the lower right block of \eqref{BlockInvW} results with
\begin{equation}
 \mat{A}^{-1} + \frac{1}{\W^{-1}_{(1,1)} } \W^{-1}_{(2:,1)} (\W^{-1}_{(2:,1)})\T = \W^{-1}_{(2:,2:)}
\end{equation}

Finally, $A^{-1}$ is extracted to be:
\begin{equation}
\mat{A}^{-1} = \W^{-1}_{(2:,2:)}- \frac{\W^{-1}_{(2:,1)} \W^{-1}_{(1,2:)}}{\W^{-1}_{(1,1)}}
\label{InvA}
\end{equation}
and the computational complexity of \eqref{InvA} is $o(d^2)$

\todo{Discuss numerical stability ?}

All of the above is summarized in Algoirthm \ref{alg:comet}

\begin{algorithm}[tb]
   \caption{COMET}
   \label{alg:comet}
\begin{algorithmic}[1]
   \STATE {\bfseries input:} training data, max number of steps, $\alpha$, $\beta$
   \STATE {\bfseries initialize:} 
   \STATE Generate a triplet set $T$, $\W  \leftarrow I_{d \times d}$ , $\W^{-1}  \leftarrow I_{d \times d}$

   \REPEAT 
   \STATE Choose a coordinate $k \in {1..d}$ randomly
   \STATE Evaluate $\mat{A}^{-1}$ using $\W^{-1}$ \eqref{InvA}
   \STATE Evaluate the coordinate step gradient matrix \eqref{gradMtx}
   \STATE Choose the step size, with an upper bound  \eqref{PDUpdateCondQuadForm}
   \STATE Update the metric to $\newW$ \eqref{updateEq}
   \STATE Update the metric inverse to $\newW^{-1}$ \eqref{InvWwdb}
   \UNTIL{max number of steps}
\end{algorithmic}
\end{algorithm}

\subsection{Computational complexity}
Our algorithm is competitively fast. Each step takes a fixed amount of operations. It keeps $\W$ within the PD cone and therefore avoid a costly projection to the PD cone following each step. Here, we evaluate the computational complexity for a single coordinate step. The complexity is comprised of an evaluation of the gradient elements and updating $\W$, $\W^{1}$ and $\mat{A}^-1$.

We evaluate the computational complexity of a single coordinate step \eqref{gradMat}. Each element $\delta_{(i,j)}$ of the gradient matrix \eqref{gradMat} equals
\begin{multline}
\delta_{(i,j)} = \sum\limits_{t\in T}{ [\tfrac{1}{2}[(\vec{q}_{t})_i(\Delta\vec{p}_{t}\T)_j + (\Delta\vec{p}_{t}\T)_i(\vec{q}_{t})_j\T] } \cdot \textbf{1}(\lambda_t)  \\ 
 - \alpha \cdot \W^{-1}_{(i,j)} + \beta \cdot \W_{(i,j)}
\label{gradMatElem}
\end{multline}

If the training data is dense, evaluating the sum in \eqref{gradMatElem} costs maximal $o(|T|)$ operations, where $|T|$ is the number of triplets. Nonetheless, if the training data is sparse, with a sparsity coefficient $\gamma$,  $ 0< \gamma <1 $, then evaluating the sum in \eqref{gradMatElem} will cost an average of $o(\gamma^2 |T|)$ operations, because we can accumulate only the elements that are both non-zeros in $(\vec{q}_{t})_i$ and in $(\Delta\vec{p}_{t}\T)_j  $ (same goes to $(\vec{q}_{t})_j$ and $(\Delta\vec{p}_{t}\T)_i$). Evaluating $\W^{-1}_{(i,j)}$ and $\W_{(i,j)}$ costs $o(1)$ since we always keep the updated versions of $\W$ and $\W^{-1}$. Evaluating all the gradient elements $\delta_{(k,:)}$ in a single row $k$ then costs $o(d\cdot \gamma^2 |T|)$. Accounting for $o(d^2)$ which is the cost of evaluating each of \eqref{InvA}, \eqref{PDUpdateCondQuadForm} and \eqref{InvWwdb}, sums up for a computational complexity cost of $o(\gamma^2 d |T| + d^2)$ per coordinate step.

Finally, the total computational complexity cost of our approach is $o(N \cdot (\gamma d)^2 |T| + N \cdot d^3)$ while taking $N \dot d$ coordinate steps, where N is the number of iterations over all the coordinates. Our empirical experimental results showed that our approach converges for choosing $N$ between 5 to 10. 

With regard to memory usage, holding the triplets costs $o(\gamma d |T|)$ elements, holding $\W$ and $\W^{-1}$ costs $o(d^2)$. The total memory usage is $o(\gamma d |T| + d^2)$
\section{Convergence}
Our method is based on minimizing a strongly convex function using block-coordinate descent. There is a well established body of work showing that when using non-overlapping blocks, block-coordinate descent iterates converge w.h.p. in a linear rate to the optimum value \cite{nesterov2012efficiency,richtarik2014iteration}.
However, the blocks we use in our method are overlapping - for example the $(1,2)$ coordinate of the matrix is a part of both the 1\textsuperscript{st} and the 2\textsuperscript{nd} column-row. We thus make use of a more general convergence result applicable to overlapping blocks, given by \citet{richtarik2013optimal}. 
In their paper, \citet{richtarik2013optimal} give sufficient conditions for linear convergence of overlapping block-coordinate descent for a strongly convex smooth objective.
In order for these conditions to hold, we must slightly modify the objective function $L({\W})$. The objective $L(\W)$ is strongly convex but not smooth, since the gradient of the log-det term is unbounded near the envelope of the positive definite cone. Let $\tilde{L}({\W}) = L({\W + \eta I_d})$, where $I_d$ is the $d \times d$ identity matrix, and $\eta$ is a fixed parameter.
Note that our algorithm can easily minimize $\tilde{L}$, the only difference being that we now need to maintain both $\W^{-1}$ and $(\W+\eta I_d)^{-1}$, which does not change the asymptotic computational complexity. The additional $\eta I_d$ term acts as a prior, where we add a Euclidean distance term to the distance we learn. 
We now show that the modified objective $\tilde{L}$ obeys Assumption $4$ and Assumption $5$ of \citet{richtarik2013optimal}.Thereby, according to \citet[Theorems 3 and 7]{richtarik2013optimal}, Algorithm \ref{alg:comet} converges to the optimum value in a linear rate.

\begin{theorem}
Let $\tilde{L}(\W)=\sum\limits_{t\in T}{l_{\W + \eta I}(\vec{q}_t, \vec{p}_{t}^{+}, \vec{p}_{t}^{-})} - \alpha \cdot \log \det(\W + \eta I) + \tfrac{1}{2} \beta \cdot \| \W + \eta I \|_{F}^{2}$, where $l_{\W + \eta I}$ is either the squared hinge loss or the log-loss, and $\tilde{L}$ is defined over the positive semidefininte cone.
Then the following two conditions hold:
\begin{enumerate}
%\item[Assumption 4, \cite{richtarik2013optimal}] the first
%\item[Assumption 5, \cite{richtarik2013optimal}] the second
\item Assumption 4, \citet{richtarik2013optimal}: $\tilde{L}$ has Lipschitz gradient with positive constant $L$, such that $\| \nabla \tilde{L}(\W) - \nabla \tilde{L}(\W + t\Z) \|_F \leq L |t|$ for all symmetric matrices $\Z \in \mathbb{R}^{d \times d}$ and $t \in \mathbb{R}$ such that $\W + t\Z$ is positive semidefinite.
\item Assumption 5, \citet{richtarik2013optimal}: 
$\tilde{L}(\W) = \sum_{S \in \mathbb{S}} \ell_S (\W)$, where $\mathbb{S}$ is a finite collection of subsets of the set of matrix index pairs $[d] \times [d]$, and $\ell_S$ are differentiable convex functions such that $\ell_S$ depends only on coordinates $(i,j) \in S$.
\end{enumerate}
\end{theorem}

\begin{proof}
proof here
\end{proof}

\begin{corollary}
Let $\W^k$ be the $k$ iterate of Algorithm \ref{alg:comet} with objective function $\tilde{L}$, and let $\tilde{L}^*$ be the optimal value of $\tilde{L}$ on the PSD cone.
If $t >$ \todo{compute what should be here}, then $Prob(\tilde{L}(\W^k) - \tilde{L}^* \leq \epsilon) \geq 1-\rho$.
\end{corollary}
\begin{proof}
proof here
\end{proof}
\section{Experiments}
%  Note that we've also tested our approach with Frobenius norm regularization, but haven't seen a significant improvement for utilizing it.
% number of steps < 10*d

\subsection{Experimental setup}
\subsection{Competing methods}
\subsection{Retrieval of similar images}
\subsection{Document retrieval}


\section{Summary}
This is the best paper ever.



% In the unusual situation where you want a paper to appear in the
% references without citing it in the main text, use \nocite
%\nocite{langley00}

\bibliography{comet}
\bibliographystyle{icml2015}

\end{document} 

